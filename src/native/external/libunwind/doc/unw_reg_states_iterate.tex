\documentclass{article}
\usepackage[fancyhdr,pdf]{latex2man}

\input{common.tex}

\begin{document}

\begin{Name}{3libunwind}{unw\_reg\_states\_iterate}{David Mosberger-Tang}{Programming Library}{unw\_reg\_states\_iterate}unw\_reg\_states\_iterate -- get register state info on current procedure
\end{Name}

\section{Synopsis}

\File{\#include $<$libunwind.h$>$}\\

\Type{int} \Func{unw\_reg\_states\_iterate}(\Type{unw\_cursor\_t~*}\Var{cp}, \Type{unw\_reg\_states\_callback}\Var{cb}, \Type{void~*}\Var{token});\\

\section{Description}

The \Func{unw\_reg\_states\_iterate}() routine provides
information about the procedure that created the stack frame
identified by argument \Var{cp}.  The \Var{cb} argument is a pointer
to a function of type \Type{unw\_reg\_states\_callback} which is used to
return the information.  The function \Type{unw\_reg\_states\_callback} has the
following definition:

\Type{int} (~*\Var{unw\_reg\_states\_callback})(\Type{void~*}\Var{token},
			\Type{void~*}\Var{reg\_states\_data},
			\Type{size\_t} \Var{reg\_states\_data\_size},
			\Type{unw\_word\_t} \Var{start\_ip}, \Type{unw\_word\_t} \Var{end\_ip});

The callback function may be invoked several times for each call of \Func{unw\_reg\_states\_iterate}. Each call is associated with an instruction address range and a set of instructions on how to update register values when returning from the procedure in that address range.  For each invocation, the arguments to the callback function are:
\begin{description}
\item[\Type{void~*} \Var{token}] The token value passed to \Var{unw\_reg\_states\_callback}. \\
\item[\Type{void~*} \Var{reg\_states\_data}] A pointer to data about
  updating register values. This data, or a copy of it, can be passed
  to \Var{unw\_apply\_reg\_state}.\\
\item[\Type{int} \Var{reg\_states\_data\_size}] The size of the register update data. \\
\item[\Type{unw\_word\_t} \Var{start\_ip}] The address of the first
  instruction of the address range.  \\
\item[\Type{unw\_word\_t} \Var{end\_ip}] The address of the first
  instruction \emph{beyond} the end of the address range.  \\
\end{description}

\section{Return Value}

On successful completion, \Func{unw\_reg\_states\_iterate}() returns
0.  If the callback function returns a nonzero value, that indicates
failure and the function returns immediately.  Otherwise the negative
value of one of the error codes below is returned.

\section{Thread and Signal Safety}

\Func{unw\_reg\_states\_iterate}() is thread safe.  If cursor \Var{cp} is
in the local address space, this routine is also safe to use from a
signal handler.

\section{Errors}

\begin{Description}
\item[\Const{UNW\_EUNSPEC}] An unspecified error occurred.
\item[\Const{UNW\_ENOINFO}] \Prog{Libunwind} was unable to locate
  unwind-info for the procedure.
\item[\Const{UNW\_EBADVERSION}] The unwind-info for the procedure has
  version or format that is not understood by \Prog{libunwind}.
\end{Description}
In addition, \Func{unw\_reg\_states\_iterate}() may return any error
returned by the \Func{access\_mem}() call-back (see
\Func{unw\_create\_addr\_space}(3libunwind)).

\section{See Also}

\SeeAlso{libunwind}(3libunwind),
\SeeAlso{unw\_apply\_reg\_state}(3libunwind)

\section{Author}

\noindent
David Mosberger-Tang\\
Email: \Email{dmosberger@gmail.com}\\
WWW: \URL{http://www.nongnu.org/libunwind/}.
\LatexManEnd

\end{document}
