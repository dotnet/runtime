\documentclass{article}
\usepackage[fancyhdr,pdf]{latex2man}

\input{common.tex}

\begin{document}

\begin{Name}{3libunwind}{unw\_init\_local}{David Mosberger-Tang}{Programming Library}{unw\_init\_local}unw\_init\_local -- initialize cursor for local unwinding
\end{Name}

\section{Synopsis}

\File{\#include $<$libunwind.h$>$}\\

\Type{int} \Func{unw\_init\_local}(\Type{unw\_cursor\_t~*}\Var{c}, \Type{unw\_context\_t~*}\Var{ctxt});\\
\Type{int} \Func{unw\_init\_local2}(\Type{unw\_cursor\_t~*}\Var{c}, \Type{unw\_context\_t~*}\Var{ctxt}, \Type{int} \Var{flag});\\

\section{Description}

The \Func{unw\_init\_local}() routine initializes the unwind cursor
pointed to by \Var{c} with the machine state in the context structure
pointed to by \Var{ctxt}.  As such, the machine state pointed to by
\Var{ctxt} identifies the initial stack frame at which unwinding
starts.  The machine state is expected to be one provided by a call to
\Func{unw\_getcontext}(); as such, the instruction pointer may point to
the instruction after the last instruction of a function, and
\Prog{libunwind} will back-up the instruction pointer before beginning
a walk up the call stack.  The machine state must remain valid for the
duration for which the cursor \Var{c} is in use.

The \Func{unw\_init\_local}() routine can be used only for unwinding in
the address space of the current process (i.e., for local unwinding).
For all other cases, \Func{unw\_init\_remote}() must be used instead.
However, unwind performance may be better when using
\Func{unw\_init\_local}().  Also, \Func{unw\_init\_local}() is
available even when \Const{UNW\_LOCAL\_ONLY} has been defined before
including \File{$<$libunwind.h$>$}, whereas \Func{unw\_init\_remote}()
is not.

If the \Type{unw\_context\_t} is known to be a signal frame (i.e.,
from the third argument in a sigaction handler on linux),
\Func{unw\_init\_local2}() should be used for correct initialization
on some platforms, passing the \Const{UNW\_INIT\_SIGNAL\_FRAME} flag.

\section{Return Value}

On successful completion, \Func{unw\_init\_local}() returns 0.
Otherwise the negative value of one of the error codes below is
returned.

\section{Thread and Signal Safety}

\Func{unw\_init\_local}() is thread safe as well as safe to use from a
signal handler.

\section{Errors}

\begin{Description}
\item[\Const{UNW\_EINVAL}] \Func{unw\_init\_local}() was called in a
  version of \Prog{libunwind} which supports remote unwinding only
  (this normally happens when calling \Func{unw\_init\_local}() for a
  cross-platform version of \Prog{libunwind}).
\item[\Const{UNW\_EUNSPEC}] An unspecified error occurred.
\item[\Const{UNW\_EBADREG}] A register needed by \Func{unw\_init\_local}()
  wasn't accessible.
\end{Description}

\section{See Also}

\SeeAlso{libunwind}(3libunwind),
\SeeAlso{unw\_init\_remote}(3libunwind)

\section{Author}

\noindent
David Mosberger-Tang\\
Email: \Email{dmosberger@gmail.com}\\
WWW: \URL{http://www.nongnu.org/libunwind/}.
\LatexManEnd

\end{document}
