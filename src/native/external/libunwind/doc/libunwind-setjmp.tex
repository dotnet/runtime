\documentclass{article}
\usepackage[fancyhdr,pdf]{latex2man}

\input{common.tex}

\begin{document}

\begin{Name}{3libunwind}{libunwind-setjmp}{David Mosberger-Tang}{Programming Library}{libunwind-based non-local gotos}libunwind-setjmp -- libunwind-based non-local gotos
\end{Name}

\section{Synopsis}

\File{\#include $<$setjmp.h$>$}\\

\noindent
\Type{int} \Func{setjmp}(\Type{jmp\_buf}~\Var{env});\\
\Type{void} \Func{longjmp}(\Type{jmp\_buf}~\Var{env}, \Type{int}~\Var{val});\\
\Type{int} \Func{\_setjmp}(\Type{jmp\_buf}~\Var{env});\\
\Type{void} \Func{\_longjmp}(\Type{jmp\_buf}~\Var{env}, \Type{int}~\Var{val});\\
\Type{int} \Func{sigsetjmp}(\Type{sigjmp\_buf}~\Var{env}, \Type{int}~\Var{savemask});\\
\Type{void} \Func{siglongjmp}(\Type{sigjmp\_buf}~\Var{env}, \Type{int}~\Var{val});\\

\section{Description}

The \Prog{unwind-setjmp} library offers a \Prog{libunwind}-based
implementation of non-local gotos.  This implementation is intended to
be a drop-in replacement for the normal, system-provided routines of
the same name.  The main advantage of using the \Prog{unwind-setjmp}
library is that setting up a non-local goto via one of the
\Func{setjmp}() routines is very fast.  Typically, just 2 or 3 words
need to be saved in the jump-buffer (plus one call to
\Func{sigprocmask}(2), in the case of \Func{sigsetjmp}).  On the
other hand, executing a non-local goto by calling one of the
\Func{longjmp}() routines tends to be much slower than with the
system-provided routines.  In fact, the time spent on a
\Func{longjmp}() will be proportional to the number of call frames
that exist between the points where \Func{setjmp}() and
\Func{longjmp}() were called.  For this reason, the
\Prog{unwind-setjmp} library is beneficial primarily in applications
that frequently call \Func{setjmp}() but only rarely call
\Func{longjmp}().

\section{Caveats}

\begin{itemize}
\item The correct operation of this library depends on the presence of
  correct unwind information.  On newer platforms, this is rarely an
  issue.  On older platforms, care needs to be taken to
  ensure that each of the functions whose stack frames may have to be
  unwound during a \Func{longjmp}() have correct unwind information
  (on those platforms, there is usually a compiler-switch, such as
  \Opt{-funwind-tables}, to request the generation of unwind
  information).
\item The contents of \Type{jmp\_buf} and \Type{sigjmp\_buf} as setup
  and used by these routines is completely different from the ones
  used by the system-provided routines.  Thus, a jump-buffer created
  by the libunwind-based \Func{setjmp}()/\Func{\_setjmp} may only be
  used in a call to the libunwind-based
  \Func{longjmp}()/\Func{\_longjmp}().  The analogous applies for
  \Type{sigjmp\_buf} with \Func{sigsetjmp}() and \Func{siglongjmp}().
\end{itemize}

\section{Files}

\begin{Description}
\item[\Opt{-l}\File{unwind-setjmp}] The library an application should
  be linked against to ensure it uses the libunwind-based non-local
  goto routines.
\end{Description}


\section{See Also}
\SeeAlso{libunwind}(3libunwind),
\SeeAlso{setjmp}(3), \SeeAlso{longjmp}(3),
\SeeAlso{\_setjmp}(3), \SeeAlso{\_longjmp}(3),
\SeeAlso{sigsetjmp}(3), \SeeAlso{siglongjmp}(3)

\section{Author}

\noindent
David Mosberger-Tang\\
Email: \Email{dmosberger@gmail.com}\\
WWW: \URL{http://www.nongnu.org/libunwind/}.
\LatexManEnd

\end{document}
